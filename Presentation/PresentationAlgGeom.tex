\documentclass{beamer}

\usepackage{amsmath}
\usepackage{amsthm}
\usepackage{amsfonts}
\usepackage{amssymb,enumerate}
\usepackage{amsthm,stmaryrd}
\usepackage[all]{xy}
\usepackage{hyperref}
\usepackage{xcolor}
\usepackage{tikz}
\usepackage{scrextend}
\usepackage{apacite}
\usetikzlibrary{shapes.geometric}


%
% Choose how your presentation looks.
%
% For more themes, color themes and font themes, see:
% http://deic.uab.es/~iblanes/beamer_gallery/index_by_theme.html
%
\mode<presentation>
{
  \usetheme{Madrid}      % or try Darmstadt, Madrid, Warsaw, ...
  \usecolortheme{beaver} % or try albatross, beaver, crane, ...
  \usefonttheme{default}  % or try serif, structurebold, ...
  \setbeamertemplate{navigation symbols}{}
  \setbeamertemplate{caption}[numbered]
  %\setbeamertemplate{theorems}[numbered]
  \setbeamercolor{structure}{fg=darkred}
   \setbeamercolor{block body}{bg=gray!10!white}

} 

\usepackage[english]{babel}
\usepackage[utf8x]{inputenc}
%\usepackage[utf8]{inputenc}
%\usepackage{enumitem}
\usepackage{amsmath}
\usepackage{amsfonts}
\usepackage{amssymb,enumerate}
\usepackage{amsthm,stmaryrd}
\usepackage{float}
\usepackage{graphicx}
\usepackage{verbatim}
\usepackage{subcaption}
\usepackage{mathtools}
\usepackage{fancyvrb}

\newtheorem{prop}{Proposition}
\newtheorem{prop1}{Proposition 1}
\newtheorem{prop2}{Proposition 2}
\newtheorem{defn}[prop]{Definition}
\newtheorem{lem}{Lemma}
\newtheorem{ex}{Example}
\newtheorem{n}{Note}
\newtheorem{cor}{Corollary}
\newtheorem{BA}{Buchberger's Algorithm}
\newtheorem{gbc1}{GB Criterion 1}
\newtheorem{gbc2}{GB Criterion 2}
\newtheorem{gbc3}{GB Criterion 3}
\newtheorem{defsnota}{Definitions and Notation}
\newtheorem{thm}{Theorem}
\newtheorem{fainf}{Facts about Ideals in Number Fields}
\newtheorem{rmk}{Remark}
\newtheorem{aoam}{Analysis of Add and Multiply}
\newtheorem{PN}{Projective Nullstellensatz}
\newtheorem{AN}{Recall Affine Nullstellensatz}
\newtheorem{IncStep}{Inclusion Step}
\newtheorem{PC}{The Partition Class}
\newtheorem{SPIP}{Small Principal Ideal Problem}
\newtheorem{cscheme}{Somewhat Homomorphic Scheme}
\newtheorem{prf}{Proof}
\usepackage{mathrsfs}


\title[Modules of Differentials]{Modules of Differentials}
\author{Alan R. Hahn}
%\institute{Technische Universit{\"a}t Kaiserslautern}
\institute{TU Kaiserslautern}
\date{May 2020}

\begin{document}

\begin{frame}
  \titlepage
\end{frame}

%%%%%%%%%%%%%%%%%%%%%%%%%%%%%%%%

\begin{frame}
\begin{small}
\begin{defn}
Let $S$ be a ring, $M$ an $S$-module. Then a map of abelian groups $d:S\to M$ is a derivation if it satisfies the Leibniz rule
$$d(fg) = fdg+gdf\text{   for }f,g\in S.$$
\end{defn}

%\begin{n}
\begin{itemize}
\item If $S$ is an R-algebra, then we say that $d$ is $R$-linear if it is a map of $R$-modules. 
\item The set Der$_R(S,M)$ of all $R$-linear derivations $S\to M$ is naturally an $S$-module, with addition the usual pointwise addition, and multiplication defined by 
$$sd:f\mapsto s(d(f))\in M.$$
for $s,f\in S$, $d\in$ Der$_R(S,M)$.
\end{itemize}
%\end{n}
\end{small}
\end{frame}

%%%%%%%%%%%%%%%%%%%%%%%%%%%%%%%%%%

\begin{frame}
\begin{small}

Recall Leibniz property of derviations: 
$$d(fg) = fdg+gdf\text{   for }f,g\in S.$$

%\begin{n}
Note that for any derivation $d$, we have d(1) = 0:
$$d(1) = d(1\cdot 1) = 1d(1)+1d(1).$$
%\end{n}

\begin{prop}
A derivation $d$ is $R$-linear iff $da = 0$ for every $a\in R$:
\\$\Rightarrow$ : If $d$ is $R$-linear, then $da = d(a\cdot 1) = ad1 = 0$.
\\$\Leftarrow$ : If $da = 0$ for all $a\in R$, then $d(as) = ads+sda = ads$.
\end{prop}


\end{small}
\end{frame}

%%%%%%%%%%%%%%%%%%%%%%%%%%%%%%%%%%%

\begin{frame}
\begin{small}
\begin{defn}[Module of K{\"a}hler Differentials]
If $S$ is an $R$-algebra, then the module of K{\"a}hler differentials of $S$ over $R$, written $\Omega_{S/R}$, is the $S$-module generated by the set $\{d(f):f\in S\}$, subject to the relations 
$$d(ss') = sd(s') + s'd(s) \hspace{7mm} (Leibniz)$$
$$d(as+a's') = ad(s)+a'd(s') \hspace{3mm} (R\text{-}linearity)$$
for all $a,a'\in R$, $s,s'\in S$.
\end{defn}

%\begin{n}
\begin{itemize}
\item We often write df for d(f)
\item The map $d:S\to\Omega_{S/R},\hspace{2mm} f\mapsto df$ is an R-linear derivation, and is called the universal R-linear derivation. 
\item The map $d$ has the following universal property, which determines it and $\Omega_{S/R}$ uniquely: Given any $S$-module $M$ and $R$-linear derivation $e:S\to M$, there is a unique $S$-linear homomorphism $e':\Omega_{S/R}\to M$ such that $e = e'd$, as in the figure to the right. Indeed, $e'$ is defined by $e'(df) := ef$.
\item Asserting the universal property is the same as asserting that 
$$Der_R(S,M)\cong Hom_S(\Omega_{S/R}, M)$$
naturally, as functors of M.
\end{itemize}
%\end{n}
\end{small}
\end{frame}

%%%%%%%%%%%%%%%%%%%%%%%%%%%%%%%%%%%

\begin{frame}
\begin{small}
%\begin{itemize}
%\item
Note that if $S$ is generated as an $R$-algebra by elements $f_i$, then $\Omega_{S/R}$ is generated as an $S$-module by the elements $df_i$: If $g = p(f_1,...,f_r)$ is a polynomial in the $f_i$ with coefficients in $R$, then repeated use of the Leibniz rule allows us to express $dg$ as an S-linear combination of the $df_i$.
\\\indent
\\In particular, if $S$ is finitely generated as an $R$-algebra, then so is $\Omega_{R/S}$.

\begin{prop}
If $S = R[x_1,...,x_r]$, the polynomial ring in $r$ variables, then $\Omega_{S/R} = \oplus_{i=1}^rSdx_i$, the free module on the $dx_i$.
\end{prop}
%\end{itemize}

\end{small}
\end{frame}

%%%%%%%%%%%%%%%%%%%%%%%%%%%%%%%%%%%

\begin{frame}
\begin{small}
\begin{n}
The association of an R-algebra S to the S-module $\Omega_{S/R}$ and the derivation $d:S\to\Omega_{S/R}$ is a functor in the following sense: 
\\Given a commutative diagram of rings as to the right, which we may regard as a morphism of pairs $\phi:(R,S)\to(R',S')$, we get an induced `morphism' as in the figure to the right, where the bottom horizontal map is the given morphism of $R$-algebras, and the upper horizontal map is a morphism of $S$-modules, obtained from the universal property of $\Omega_{S/R}$ applied to the $R$-linear derivation $S\to\Omega_{S'/R'}$ that is the composition of $S\to S$' with $S'\to\Omega_{S'/R'}$. 
\end{n}

\begin{itemize}
\item Sometimes, the $S$-linear map $\Omega_{S/R}\to\Omega_{S'/R'}$ is replaced with the equivalent data of the $S$'-linear map $S'\otimes_S\Omega_{S/R}\to\Omega_{S'/R'}$.
\end{itemize}

\end{small}
\end{frame}

%%%%%%%%%%%%%%%%%%%%%%%%%%%%%%%%%%%

\begin{frame}
\begin{small}
\begin{prop}[Relative Cotangent Sequence]
If $R\to S\to T$ are maps of rings, then there is a right exact sequence of $T$-modules 
$$T\otimes_S\Omega_{S/R}\to\Omega_{T/R}\to\Omega_{T/S}\to0$$
where the right-hand map takes $dc$ to $dc$, and the left-hand map takes $c\otimes db$ to $cdb$. 
\end{prop}


\begin{prf}
The set of generators $\{d(t):t\in T\}$ is the same for both $\Omega_{T/S}$ and $\Omega_{T/R}$, the only difference is that in $\Omega_{T/S}$ there are extra relations of the form $db = 0$ for $b\in S$. Note that $\{db : b\in S\}$ is exactly the image of the generators $1\otimes db$ of $T\otimes_S\Omega_{S/R}$. 
\end{prf}


\begin{itemize}
\item Note that when $S\to T$ is an epimorphism, $\Omega_{T/S}$ is 0 by $S$-linearity: $dc = 0$ with $c$ in the image of $S$, by $S$-linearity. 
\end{itemize}


\end{small}
\end{frame}

%%%%%%%%%%%%%%%%%%%%%%%%%%%%%%%%%%%



\begin{frame}
\begin{footnotesize}


\begin{itemize}
\item Note that when $S\to T$ is an epimorphism, $\Omega_{T/S}$ is 0 by $S$-linearity: $dc = 0$ with $c$ in the image of $S$, by $S$-linearity. 
\end{itemize}


\begin{prop}[Conormal Sequence]
If $\pi:S\to T$ is an epimorphism of $R$-algebras with kernel $I$, then there is an exact sequence of $T$-modules
$$I/I^2\xrightarrow[\text{}]{d}T\otimes_S\Omega_{S/R}\xrightarrow[\text{}]{D\pi}\Omega_{T/R}\xrightarrow{}{} 0$$
where the right-hand map is given by $D_{\pi}:c\otimes db\mapsto cdb$ and the left-hand map takes the class of $f$ to $1\otimes df$.
\end{prop}

\begin{prf}
Consider $d:I\to\Omega_{S/R}$ that is the restriction of the universal derivation $S\to\Omega_{S/R}$. If $b\in S$ and $c\in I$, then the Leibniz formula $d(bc) = bdc+cdb$ shows that $d$ induces an $S$-linear map $I\to (\Omega_{S/R})/(I\Omega_{S/R}) = T\otimes_S\Omega_{S/R}$. Taking $b\in I$ as well, we see also that $I^2$ goes to 0 in $T\otimes_S\Omega_{S/R}$, so we get a map of $T$-modules $d:I/I^2\to T\otimes_S\Omega_{S/R}.$
\\\indent 
\\$T\otimes_S\Omega_{S/R}$ is generated as a T-module by $\{db:b\in S\}$ subject to the relations of $R$-linearity and the Leibniz rule. This is the same as the description by generators and relations of $\Omega_{T/R}$, except that in $\Omega_{T/R}$ the elements $df$ for $f\in I$ are $df = d0 = 0$ as $0\in R$. Thus im($d$) = ker($D_\pi$).
\end{prf}


\end{footnotesize}
\end{frame}







%%%%%%%%%%%%%%%%%%%%%%%%%%%%%%%%%%5%%

\begin{frame}
\begin{small}
\begin{ex}[Computation of Differentials]
%If $\pi:S\to T$ is an epimorphism of $R$-algebras with kernel $I$, then we have: 

%$$I/I^2\xrightarrow[\text{}]{d}T\otimes_S\Omega_{S/R}\xrightarrow[\text{}]{D\pi}\Omega_{T/R}\xrightarrow{}{} 0.$$]

%$$I/I^2\xrightarrow[\text{}]{d}R[x_1,...,x_r]/I\otimes_{R[x_1,...,x_r]}\Omega_{R[x_1,...,x_r]/R}\xrightarrow[\text{}]{D\pi}\Omega_{(R[x_1,...,x_r]/I)/R}\xrightarrow{}{} 0.$$

%Note $R[x_1,...,x_r]/I\otimes_{R[x_1,...,x_r]}\Omega_{R[x_1,...,x_r]/R} = R[x_1,...,x_r]/I\otimes_{R[x_1,...,x_r]}(\oplus_{i=1}^r(R[x_1,...,x_r])dx_i)$

%So that $\Omega_{(R[x_1,...,x_r]/I)/R}\cong$ coker$(I/I^2\xrightarrow[\text{}]{d}R[x_1,...,x_r]/I\otimes_{R[x_1,...,x_r]}\Omega_{R[x_1,...,x_r]/R})$

Consider $\pi:S\to T$ with $S = R[x_1,....,x_r], T = R[x_1,...,x_r]/I$,  as in the previous proposition. Then we have
$$I/I^2\xrightarrow[\text{}]{d}T\otimes_S\Omega_{S/R}\xrightarrow[\text{}]{D\pi}\Omega_{T/R}\xrightarrow{}{} 0.$$

Note in this case that $T\otimes_{S}\Omega_{S/R} = (R[x_1,...,x_r]/I)\otimes_{S}(\oplus_{i=1}^rSdx_i)$
$$= (\oplus_{i=1}^r(R[x_1,...,x_r]/I)dx_i) = \oplus_{i=1}^r(Tdx_i).$$

From the conormal sequence, we see that 
$$\Omega_{T/R}\cong\text{coker }(d:I/I^2\to T\otimes_{S}\Omega_{S/R}) = \text{coker }(d:I/I^2\to \oplus_{i=1}^r(Tdx_i)).$$

Writing $I/I^2$ as a homomorphic image of a free $T$-module with generators $e_i$ going to the classes of the $f_i$, the composition
$$\mathcal{J}:\oplus Te_i\twoheadrightarrow I/I^2\to\oplus_{i=1}^rTdx_i$$
is a map of free T-modules that is represented by the Jacobian matrix of the $f_j$ with respect to the $x_i$.

\end{ex}

\end{small}
\end{frame}


%%%%%%%%%%%%%%%%%%%%%%%%%%%%%%%%%%%

\begin{frame}
\begin{small}

\begin{ex}[Computation of Differentials, continued]
Recall $\pi:S\to T$ with $S = R[x_1,....,x_r], T = R[x_1,...,x_r]/I$, and we had that 
$$\Omega_{T/R}\cong \text{coker }(d:I/I^2\to \oplus_{i=1}^r(Tdx_i)).$$

Then we noted there is a map
$$\mathcal{J}:\oplus Te_i\twoheadrightarrow I/I^2\to\oplus_{i=1}^rTdx_i$$ sending $e_i$ to $f_i$ and is a map of free $T$-modules that is represented by the Jacobian matrix of the $f_j$ with respect to the $x_i$.
\\\indent
\\In short, we have that 
$$\Omega_{T/R}\cong \text{coker }(d:I/I^2\to \oplus_{i=1}^r(Tdx_i)) \cong \text{coker }(\mathcal{J} = (\partial f_j/\partial x_i)).$$

\end{ex}


\end{small}
\end{frame}



%%%%%%%%%%%%%%%%%%%%%%%%%%%%%%%%%%%

\begin{frame}
\begin{small}
Recall that in our setting we have that: 
$$\Omega_{T/R}\cong \text{coker }(d:I/I^2\to \oplus_{i=1}^r(Tdx_i)) \cong \text{coker }(\mathcal{J} = (\partial f_j/\partial x_i)).$$

%\begin{ex}[Explicit examples]
%\begin{itemize}
\underline{\textbf{Explicit examples}}: 
\\\indent 
\\$-$ If $T = R[x]/f(x)$, then 
$$\Omega_{T/R} = Tdx/(df) =  Tdx/im(d) = Tdx/im(\mathcal{J}) = Tdx/(T\cdot f'(x)dx)\cong T/(f'(x)).$$
\\\indent
\\$-$ If $T = R[x,y,t]/(y^2-x^2(t^2-x))$, then 
\\$$\mathcal{J} = 
\begin{pmatrix}
\partial f/\partial x\\
\partial f/\partial y\\
\partial f/\partial t
\end{pmatrix}
 = 
 \begin{pmatrix}
3x^2-2xt^2\\
2y\\
-2x^2t
\end{pmatrix}$$
and $\Omega_{T/R}$ is the free $T$-module on the $dx,dy,dt$ modulo the relation
%$$\Omega_{T/R} = Sdx\oplus Sdy\oplus Sdz,$$
%modulo the relation 
$$(3x^2-2xt^2)dx + (2y)dy + (-2x^2t)dt = 0.$$
%\end{itemize}
%\end{ex}

\end{small}
\end{frame}



%%%%%%%%%%%%%%%%%%%%%%%%%%%%%%%%%%%

\begin{frame}
\begin{small}
\underline{An aside about manifolds}:
\\\indent
\\If $Y$ is an affine algebraic variety over a field $k$ with coordinate ring $S$, then $\Omega_{S/k}$ plays the role of the cotangent bundle of $Y$. More generally, if $Y\to X$ is a morphism of affine varieties corresponding to a map $R\to S$ of coordinate rings, then $\Omega_{S/R}$ plays the role of the relative cotangent bundle of this map:  
\\\indent
\\ For every smooth manifold $X$ (over say $k = \mathbb{R}$), there is a vector bundle on $X$, called the tangent bundle of $X$ and written $T_X$, whose fiber over a point $x\in X$ is the tangent space $T_{X,x}$ of $X$ at $x$. If $\varphi:X\to Y$ is a differentiable map of smooth manifolds, then for every $x\in X$, the derivative of $\varphi$ is a map $T_{\varphi} :T_{X,x}\to T_{Y,\varphi(x)}$; these derivatives fit nicely together into a map of vector bundles on $X$
$$T\varphi: T_X\to \varphi^{*}T_y,$$
where $\varphi^{*}T_y$ is the tangent bundle to $Y$ ``pulled back'' along $\varphi$ (the pullback may be defined as the fiber product $X\times_YT_Y$).




\end{small}
\end{frame}

%%%%%%%%%%%%%%%%%%%%%%%%%%%%%%%%%%


\begin{frame}
\begin{small}

Now, let $S'$ be the ring of smooth functions on $X$. For any $f\in S'$, thought of as a mapping to the line $\mathbb{R}$, the derivative $Tf:T_X\to \varphi^{*}T_\mathbb{R} = X\times\mathbb{R}$ is a linear functional on each fiber $T_{X,x}$ that varies smoothly with $x$. Thus, $Tf$ may be considered to be a global section of the dual $T^{*}_X$ of the tangent bundle, which is called the cotangent bundle of $X$. If $g$ is another function, then $T(fg) = fTg + gTf$ so that we may think of $T$ as a derivation of the ring $S'$ of smooth functions on $X$ to the $S'$-module $\Omega'$ of global sections of the cotangent bundle of $X$. 
\\\indent
\\From the universal property of the module of K{\"a}hler differentials, it follows that there is an $S'$-module homomorphism $\alpha:\Omega_{S'/k}\to \Omega'$ carrying the universal derivation d to the derivation $T$ just constructed. This is usually not an isomorphism, but if $X$ is a real affine variety and $S$ is its coordinate ring, then it can be shown that $\Omega_{S/k}$ is the algebraic object analogous to $\Omega'$ in the sense that $\Omega' = \Omega_{S/k}\otimes_S S'$. As the bundle $T^{*}_X$ and the module $\Omega'$ are equivalent objects, we see that the algebraic module of differentials $\Omega_{S/k}$ is a good stand in for the cotangent bundle. Similarly, its dual Der$_R(S,S)$ is a satisfactory replacement for the tangent bundle. 



\end{small}
\end{frame}



%%%%%%%%%%%%%%%%%%%%%%%%%%%%%%%%%%%

\begin{frame}
\begin{small}
\begin{prop}[Base Change]
Formation of differentials commutes with arbitrary ``base change from R''; that is, for any $R$-algebras $R'$ and $S$, there is a commutative diagram as in the figure to the right: 
\end{prop}


\end{small}
\end{frame}



%%%%%%%%%%%%%%%%%%%%%%%%%%%%%%%%%%%

\begin{frame}
\begin{footnotesize}

\begin{defn}[Restricted Tensor Product]
For a (possibly infinite) set of $R$-algebras $\{S_i\}$, the restricted tensor product of the $S_i$, written $\otimes_{R}S_i$ or $\otimes_{R,i}S_i$, is the algebra generated by the symbols
$$b_1\otimes b_2\otimes\cdots \text{ with }b_i\in S_i, b_i = 1\text{ for all but finitely many } i,$$ 
modulo the relations of R-multilinearity
$$b_1\otimes\cdots\otimes(ab_i+a'b'_i)\otimes\cdots = a(b_1\otimes\cdots\otimes b_i\otimes\cdots) + a'(b_1\otimes\cdots\otimes b'_i\otimes\cdots)$$
for $a,a'\in R, b_i,b'_i\in S_i$.
\end{defn}

\begin{prop}[Tensor Products]
If $T = \otimes_{R}S_i$ is the restricted tensor product of some $R$-algebras $S_i$, then 
$$\Omega_{T/R}\cong\oplus_i(T\otimes_{S_i}\Omega_{S_i/R}) = \oplus_i((\otimes_{R,j\neq i}S_j)\otimes_R\Omega_{S_i/R})$$
by an isomorphism satisfying 
$$\alpha,\hspace{2mm} d(\cdots\otimes 1\otimes 1\otimes b_i\otimes 1\otimes 1\cdots)\mapsto (...,0,0,1\otimes db_i,0,0,...) $$
where $b_i\in S_i$ occurs in the $i$th place in each expression.
\end{prop}

\end{footnotesize}
\end{frame}


%%%%%%%%%%%%%%%%%%%%%%%%%%%%%%%%%%%

\begin{frame}
\begin{small}

\begin{cor}
If $T:=S[x_1,...,x_r]$ is a polynomial ring over an $R$-algebra $S$, then 
$$\Omega_{T/R}\cong (T\otimes_S\Omega_{S/R})\oplus(\oplus_iTdx_i).$$
\end{cor}

\begin{defn}
Given a pair of maps between R-algebras $\psi,\psi':S_1\to S_2$, the coequalizer of the pair of maps is the algebra $T = S_2/I$, where $I$ is the ideal generated by all elements of the form $\psi(b)-\psi'(b)$ for $b\in S_1$.
\end{defn}

\begin{cor}[Coequalizers]
Formation of differentials preserves coequalizers in the following sense: If $T$ is the coequalizer of a pair of maps between R-algebras $\psi,\psi':S_1\to S_2$, then there is a right exact sequence of $T$-modules

$$T\otimes_{S_1}\Omega_{S_1/R}\xrightarrow[\text{}]{T\otimes D\psi - T\otimes D\psi'} T\otimes_{S_2}\Omega_{S_2/R}\to \Omega_{T/R}\to 0.$$

\end{cor}



\end{small}
\end{frame}



%%%%%%%%%%%%%%%%%%%%%%%%%%%%%%%%%%%

\begin{frame}
\begin{small}

\begin{thm}[Colimits]
Let $\mathcal{B}$ be a diagram in the category of $R$-algebras. Set $\lim\limits_{\rightarrow} \mathcal{B} = T$. If $F$ is the functor from $\mathcal{B}$ to the category of $T$-modules taking an object $S$ to $T\otimes_S\Omega_{S/R}$ and a morphism $\phi:S\to S'$ to the morphism $1\otimes D\phi: T\otimes_{S'}\Omega_{S'/R}\to T\otimes_S\Omega_{S/R}$, then 
$$\Omega_{T/R} = \lim_{\to} F.$$
\end{thm}

\begin{itemize}
\item In the category of $R$-algebras, the coproduct of a (possibly infinite) set of algebras $\{S_i\}$ is the restricted tensor product $\otimes_RS_i$ of the $S_i.$

\end{itemize}

\begin{prf}
As colimits are constructed from coproducts and coequalizers, it is enough to check the proposition for each of these two types of colimits. The case of coproducts is handled by Proposition (Tensor Products), while the case of coequalizers is handled by Corollary (Coequalizers).
\end{prf}

\begin{itemize}
\item The formation of the module of differentials does not commute with inverse limits in general.
\end{itemize}

\end{small}
\end{frame}



%%%%%%%%%%%%%%%%%%%%%%%%%%%%%%%%%%%

\begin{frame}
\begin{footnotesize}

\begin{prop}[Localization]
Formation of differentials commutes with localization of the upper argument; that is that if $S$ is an $R$-algebra and $U$ is a multiplicatively closed subset of $S$, then 
$$\Omega_{S[U^{-1}]/R}\cong S[U^{-1}]\otimes_S\Omega_{S/R}$$
in such a way that $d(1/s) = -s^{-2}ds$ for $s\in U$. 

\end{prop}

\begin{prf}
Suppose first that $U:=\{s^r:r\in\mathbb{N}\}$ for some $s\in S$ so that $S[U^{-1}] = S[x]/\langle sx-1\rangle$. By the computation to the right, we have that 
$$\Omega_{S[U^{-1}]/R} = (S[U^{-1}]\otimes_S\Omega_{S/R}\oplus S[U^{-1}]dx)/(S[U^{-1}]d(sx-1)).$$
\\Note $d(sx-1) = sdx + xds$ with $s$ a unit in $S[U^{-1}]$ so that by the further computation to the right we have that 
$$\Omega_{S[U^{-1}]/R} = S[U^{-1}]\otimes_S\Omega_{S/R}$$ 
as claimed. Also, by the relation $sdx+xds = 0$, we may identify $dx$ with $(-x/s)ds$ and noting that $x = s^{-1}$ we have that $ds^{-1} = dx = (-x/s)ds = (-{s^{-1}}/s)ds = -s^{-2}ds$ also as claimed. 
%\\To finish, if $\mathcal{B}$ is the diagram of $R$-algebras whose objects are the localizations $S[s^{-1}]$ for $s\in U$ with maps $S[s^{-1}]\to S[(st)^{-1}]$ the natural localization maps for $s,t\in U$, then $S[U^{-1}] = \lim\limits_{\rightarrow} \mathcal{B}$, so that we have by Theorem (Colimits)
%$$\Omega_{S[U^{-1}]/R} = \lim\limits_{\rightarrow}_{s\in U} S[U^{-1}]\otimes_{S[s^{-1}]}\Omega_{S[s^{-1}]/R}$$




\end{prf}



\end{footnotesize}
\end{frame}

\begin{frame}
\begin{small}
\begin{prop}[Direct Products] If $S_1, ..., S_n$ are $R$-algebras, then 
$$\Omega_{(\prod_iS_i)/R} = \prod_i\Omega_{S_i/R}.$$
\end{prop}

\begin{prf}
If $e_i$ is the idempotent of $\prod_i S_i$ that is the unit of $S_i$, and $D$ is a derivation of $\prod_i S_i$ to a $(\prod_i S_i)$-module M, then it is a fact that $De_i = 0$, so 
$$D(e_if) = e_iDf + fD(e_i) = e_iD(f).$$
Thus $D$ maps $S_i = e_i(\prod_i S_i)$ to $e_i(M) := M_i$ and corresponds to a unique map $\Omega_{S_i/R}\to M_i$. It follows that $\prod_i\Omega_{S_i/R}$ has the universal property that characterizes $\Omega_{(\prod_iS_i)/R}$.

\end{prf}

\end{small}
\end{frame}









\end{document}